\documentclass[UTF8]{ctexbook}

\usepackage[]{ctex}
\usepackage{geometry}
\usepackage[colorlinks=true]{hyperref}
\usepackage{indentfirst}
\usepackage[center]{titlesec}
\usepackage{graphicx}

\geometry{left=1cm,right=1cm,top=3cm,bottom=3cm}
\title{db的日常笔记}
\date{\today}
\author{dbydd}    
\kaishu
\setlength{\parindent}{2em}
\titleformat{\section}[block]{\LARGE\itshape\mdseries}{\arabic{section}}{1em}{}[]
\titleformat{\subsection}[block]{\Large\itshape\mdseries}{\arabic{section}.\arabic{subsection}}{1em}{}[]
\titleformat{\subsubsection}[block]{\large\itshape\mdseries}{\arabic{section}.\arabic{subsection}.\arabic{subsubsection}}{1em}{}[]
\titleformat{\paragraph}[block]{\small\bfseries}{[\arabic{paragraph}]}{1em}{}[]
\setcounter{secnumdepth}{3}
\setcounter{tocdepth}{2}

\newcommand{\limNormal}[1]{$\lim\limits_{#1}$}
\newcommand{\myLimToZero}{\limNormal{x \to 0}}
\newcommand{\myLimToInf}{\limNormal{x \to \infty}}
\newcommand{\mathCombination}[2]{C_{#1}^{#2}}
\newcommand{\mathPermutation}[2]{P_{#1}^{#2}}
\newcommand{\UpDownSum}[2]{$\sum_{#1}^{#2}$}


\begin{document}
\pagestyle{empty}{
  \maketitle
  \paragraph{todos}{
    标签:求导,泰勒公式
  }
  \tableofcontents
  \newpage
}
\setcounter{page}{1}
\chapter{数学}{
\section{基本概念}{
  \subsection{六大基本初等函数}{
    常数函数,幂函数,指数函数,对数函数,三角函数
  }

  \subsection{二项式定理}{
    $(x + y)^n = x^n + \mathCombination{n - 1}{n}(x^{n-1} y) + \mathCombination{n - 2}{n}(x^{n-2} y^2) + \dots + y^n$
  }

  \subsection{排列组合}{
    排列:$\mathPermutation{m}{n} = \frac{m!}{(m-n)!}$

    组合:$\mathCombination{m}{n} = \frac{\mathPermutation{m}{n}}{m!} = \frac{n!}{m!(n-m)!}$
  }

  \subsection{零散的定义}{
    \begin{enumerate}
      \item 有界:$\exists\epsilon,f(x) < \epsilon\quad(-\infty < x < \infty )$
    \end{enumerate}
  }

  \subsection{零散的思想}{
    \begin{enumerate}
      \item 正变换是数学的重要工具,三角变换是只变其形不变其质的。三角变换常常先寻找式子所包含的各个角之间的联系,并以此为依据选择可以联系它们的适当公式,通过换元法把三角恒等变换问题转化为代数恒等变换问题。
    \end{enumerate}
  }
 }

\section{三角函数}{
三角函数一般由单位圆引出,如下:

\includegraphics{resources/UnitCircle.png}

\subsection{正三角函数}{
  $\sin{\alpha} = \frac{y}{r}$

  $\cos{\alpha} = \frac{x}{r}$

  $\tan{\alpha} = \frac{y}{x}$
}
\subsection{反三角函数}{
  $\cot{\alpha} = \frac{1}{\tan{\alpha}}$

  $\sec{\alpha} = \frac{1}{\cos{\alpha}}$

  $\csc{\alpha} = \frac{1}{\sin{\alpha}}$
}

\subsection{和差化积}{
  $\sin{\alpha}+\sin{\beta} = 2\sin{\frac{\alpha + \beta}{2}}\cos{\frac{\alpha - \beta}{2}}$

  $\cos{\alpha}+\cos{\beta} = 2\cos{\frac{\alpha + \beta}{2}\cos{\frac{\alpha-\beta}{2}}}$

  $\cos{\alpha}-\cos{\beta} = -2\sin{\frac{\alpha + \beta}{2}}\cos{\frac{\alpha - \beta}{2}}$

  $\sin{\alpha}-\sin{\beta} = 2\sin{\frac{\alpha + \beta}{2}}\cos{\frac{\alpha - \beta}{2}}$

  $\tan\alpha - \tan\beta = \tan(\alpha - \beta) \cdot (1 + \tan\alpha\tan\beta)$
}

\subsection{积化和差}{
  $\cos(\alpha + \beta) = \cos{\alpha}\cos{\beta} - \sin{\alpha}\sin{\beta}$

  $\cos(\alpha - \beta) = \cos{\alpha}\cos{\beta} + \sin{\alpha}\sin{\beta}$

  $\sin(\alpha \pm \beta) = \sin{\alpha}\cos{\beta} \pm \cos{\alpha}\sin{\beta}$

  $\tan(\alpha + \beta) = \frac{\tan\alpha + \tan\beta}{1 - \tan\alpha\tan\beta}$

  $\tan(\alpha - \beta) = \frac{\tan\alpha - \tan\beta}{1 + \tan\alpha\tan\beta}$

  $\sin{\alpha}\cos{\beta} = \frac{1}{2}[\sin{(\alpha + \beta)} + \sin{(\alpha - \beta)}]$

  $\cos{\alpha}\cos{\beta} = \frac{1}{2}[\cos{(\alpha + \beta)} + \cos{(\alpha - \beta)}]$

  $\sin{\alpha}\sin{\beta} = -\frac{1}{2}[\cos{(\alpha + \beta)} - \cos{(\alpha - \beta)}]$
}

\subsection{诱导公式}{
  \indent 奇变偶不变,符号看象限。
  \subsubsection{第一组诱导公式}{
    $\sin{(2k\pi + \alpha)} = \sin{\alpha}$

    $\cos{(2k\pi + \alpha)} = \cos{\alpha}$

    $\tan(2k\pi + \alpha) = \tan\alpha$

    $\cot(2k\pi + \alpha) = \cot\alpha$
  }

  \subsubsection{第二组诱导公式}{
    $\sin(-\alpha) = -\sin\alpha$

    $\cos(-\alpha) = \cos\alpha$

    $\tan(-\alpha) = -\tan\alpha$

    $\cot(-\alpha) = -\cot\alpha$
  }

  \subsubsection{第三组诱导公式}{
    $\sin(\pi + \alpha) = -\sin\alpha$

    $\cos(\pi + \alpha) = -\cos\alpha$

    $\tan(\pi + \alpha) = \tan\alpha$

    $\cot(\pi + \alpha) = \cot\alpha$
  }

  \subsubsection{第四组诱导公式}{
    $\sin(\pi - \alpha) = \sin\alpha$

    $\cos(\pi - \alpha) = -\cos\alpha$

    $\tan(\pi - \alpha) = -\tan\alpha$

    $\cot(\pi - \alpha) = -\cot\alpha$
  }

  \subsubsection{第五组诱导公式}{
    $\sin(\frac{\pi}{2} - \alpha) = \cos\alpha$

    $\cos(\frac{\pi}{2} - \alpha) = \sin\alpha$

    $\tan(\frac{\pi}{2} - \alpha) = \cot\alpha$

    $\cot(\frac{\pi}{2} - \alpha) = \tan\alpha$
  }

  \subsubsection{第六组诱导公式}{
    $\sin(\frac{\pi}{2} + \alpha) = \cos\alpha$

    $\cos(\frac{\pi}{2} + \alpha) = -\sin\alpha$

    $\tan(\frac{\pi}{2} + \alpha) = -\cot\alpha$

    $\cot(\frac{\pi}{2} + \alpha) = -\tan\alpha$
  }

  \subsubsection{杂项}{
    $a\sin\alpha + b\cos\alpha = \sqrt{a^2 + b^2}\sin(\alpha+\beta)$

    $\cos\alpha = 2cos^2\frac{\alpha}{2} - 1 = 1-2\sin^2\frac{\alpha}{2}$
  }

}

\subsection{倍角公式}{
\subsubsection{半倍角公式}{
  $\sin\frac{\alpha}{2} = \pm\sqrt{\frac{1 - \cos\alpha}{2}}$

  $\cos\frac{\alpha}{2} = \pm\sqrt{\frac{1 + \cos\alpha}{2}}$

  $\tan\frac{\alpha}{2} = \pm\sqrt{\frac{1-\cos\alpha}{1+\cos\alpha}} = \frac{\sin\alpha}{1+\cos\alpha} = \frac{1-\cos\alpha}{\sin\alpha}$

  $\cot\frac{\alpha}{2} = \frac{1+\cos\alpha}{\sin\alpha} = \frac{\sin\alpha}{1-\cos\alpha}$

  $\sec\frac{\alpha}{2} = \frac{\pm\sqrt{\frac{\sec\alpha - 1}{2\sec\alpha}}2\sec\alpha}{\sec\alpha + 1} = \frac{\pm\sqrt{\frac{4\sec^3\alpha + \sec^2\alpha}{2\cos\alpha}}}{\sec\alpha + 1}$

  $\csc\frac{\alpha}{2} = \frac{\pm\sqrt{\frac{\sec\alpha - 1}{2\sec\alpha}}2\sec\alpha}{\sec\alpha - 1} = \frac{\pm\sqrt{\frac{3\sec^3\alpha - \sec^2\alpha}{2\sec\alpha}}}{\sec\alpha - 1}$
}

\subsubsection{二倍角公式}{
  $\sin2\alpha = 2\sin\alpha\cos\alpha$

  $\cos2\alpha = \cos^2\alpha - \sin^\alpha = 2\cos^2\alpha - 1 = 1 - 2\sin^2\alpha$

  $\tan2\alpha = \frac{2\tan\alpha}{1 - \tan^2\alpha}$
}

\subsubsection{n倍角公式}{
$\cos{n\theta} = \sum_{i = 0}^{\frac{n}{2}}[(-1)^i\mathCombination{2i + 1}{n}\cos^{n - 2i}\theta\sin^{2i}\theta]$

$\sin{n\theta} = \sum_{i = 0}^{\frac{n}{2}}[(-1)^i\mathCombination{2i + 1}{n}\cos^{n - 2i - 1}\theta\sin^{2i+1}\theta]$
}

\subsubsection{万能替换公式}{
  角$\alpha(\alpha \neq 2k\pi + \pi ,k \in \mathbf{z})$的所有三角比都可以用 $\tan\frac{\alpha}{2}$表示.这组公式叫做万能替换公式

  $\sin\alpha = \frac{2\tan\frac{\alpha}{2}}{1+\tan^2\frac{\alpha}{2}}$

  $\cos\alpha = \frac{1 - \tan^2\frac{\alpha}{2}}{1 + \tan^2\frac{\alpha}{2}}$

  $\tan\alpha \frac{2\tan\frac{\alpha}{2}}{1 - \tan^2\frac{\alpha}{2}}$
}

\subsection{三角恒等式}{
$\csc^2{x} - cot^2{x} = 1$

$\sec^2x - tan^2x = 1$ %去你🐎的大括号

$\sin^2x + \cos^2x = 1$

$\tan{x} = \frac{\sin{x}}{\cos{x}}$
}

}

\section{微积分}{

  \subsection{极限}{

    \subsubsection{定理}{
      \begin{enumerate}
        \item 函数在一点极限存在的条件是左右极限存在且相等
        \item 洛必达法则:当极限为$\frac{0}{0}$或者$\frac{\infty}{\infty}$时可上下同时求导,求导后极限不变,每一步都需要重新判断是否依然符合类型
      \end{enumerate}
    }

    \subsubsection{重要极限}{
      \myLimToZero$\frac{\sin{x}}{x}=1$ $\to$ \limNormal{x \to 0}$\sin{x} \to x$

      \myLimToInf$(1+\frac{1}{x})^x = e$
    }
  }

  \subsection{等价无穷小}{
    \myLimToZero$a^x - 1 \approx x\ln{a}$

    \myLimToZero$\arcsin(a)x \approx \sin(a)x \approx (a)x$

    \myLimToZero$\arctan(a)x \approx \tan(a)x \approx (a)x$

    \myLimToZero$\ln1+x \approx x$

    \myLimToZero$e^x \approx 1+x$

    \myLimToZero$\sqrt{1 + x} - \sqrt{1 - x} \approx x$

    \myLimToZero$\tan{x} \approx x$

    \myLimToZero$(1 + ax)^b - 1 \approx abx$

    \myLimToZero$(1+x)^\alpha \approx 1+\alpha x$

    \myLimToZero$1 - \cos x \approx \frac{x^2}{2}$

    \myLimToZero$x - \ln(1 + x) \approx \frac{x^2}{2}$

    \myLimToZero$\tan x - x \approx \frac{x^3}{3}$

    \myLimToZero$x - \arctan x \approx \frac{x^3}{3}$

    \myLimToZero$x - \sin x \approx \frac{x^3}{6}$

    \myLimToZero$\arcsin x - x \approx \frac{x^3}{6}$

    以上等价无穷小都可以由泰勒公式推出
  }

 }

\end{document}