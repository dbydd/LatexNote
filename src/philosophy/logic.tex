\chapter{逻辑学}{

\chapter{命题公式}{

  \section{连结词}{

    \begin{itemize}
      \item $P,Q,R,...$表示原子命题(或简单命题)
      \item $1$ 表示命题的真值为真,$0$ 表示命题的真值为假
      \item $\lnot$(非)称为否定连结词.
      \item {
            $\lnot P$称为$P$的否定式 :

            \begin{center}
              \begin{tabular}{c|c}
                \hline
                $P$ & $\lnot P$ \\
                \hline
                $0$ & $1$       \\
                $1$ & $0$       \\
                \hline
              \end{tabular}
            \end{center}

            否定的定义就是取反,如果被连结词是0那么就取1,如果是1那么就取0.
            }
      \item $\land$(与)称为合取连结词
      \item {
            $P \land Q$称为$P$与$Q$的合取式 :

            \begin{center}
              \begin{tabular}{c|c|c}
                \hline
                $P$ & $Q$ & $P \land Q$ \\
                \hline
                0   & 0   & 0           \\
                0   & 1   & 0           \\
                1   & 0   & 0           \\
                1   & 1   & 1           \\
                \hline
              \end{tabular}
            \end{center}

            合取的定义就是只有当两边都是1的时候才取1.

            注意左边两列,可以当成二进制读出"0","1","2","3",此表称为真值表.

            构造真值表的方法就是将所有命题列出,每个命题一列,当有$n$个命题时就会有$2^n$行,从左往右第一个命题二分,一半是0一半是1,第二列在第一列的基础上再次二分$\dots$,如此就能写出所有取值.
            }
      \item $\lor$(或)称为析取连结词
      \item {
            $P \lor Q$称为$P$与$Q$的析取式 :

            \begin{center}
              \begin{tabular}{c|c|c}
                \hline
                $P$ & $Q$ & $P \lor Q$ \\
                \hline
                0   & 0   & 0          \\
                0   & 1   & 1          \\
                1   & 0   & 1          \\
                1   & 1   & 1          \\
                \hline
              \end{tabular}
            \end{center}

            析取的定义就是只有当两边都是0时才取0,否则取1.
            }
      \item $\to$(如果$\dots$则$\dots$)称为蕴涵连结词
      \item {
            $P \to Q$称为$P$与$Q$的蕴涵式 :

            \begin{center}
              \begin{tabular}{c|c|c}
                \hline
                $P$ & $Q$ & $P \to Q$ \\
                \hline
                0   & 0   & 1         \\
                0   & 1   & 1         \\
                1   & 0   & 0         \\
                1   & 1   & 1         \\
              \end{tabular}
            \end{center}

            蕴含的定义就是命题的推导 : 如果$P$是个假命题(0),$Q$也是个假命题(0),那么"P的结论是Q"就是个真命题($P \to Q \Leftrightarrow 1$)

            只考虑形式逻辑不考虑内在联系,因此有局限性,有时候会得到一些奇怪且显然错误的结论.

            可以把它当成$\leq$(?)
            }
      \item $\leftrightarrow$ 称为等价联结词
      \item {
            $P \leftrightarrow Q$称为$P$与$Q$的等价式 :

            \begin{center}
              \begin{tabular}{c|c|c}
                \hline
                $P$ & $Q$ & $P \leftrightarrow Q$ \\
                \hline
                0   & 0   & 1                     \\
                0   & 1   & 0                     \\
                1   & 0   & 0                     \\
                1   & 1   & 1                     \\
                \hline
              \end{tabular}
            \end{center}

            只有当两边命题都是真命题或者假命题的时候才是真.
            }
    \end{itemize}

    联结词优先级为 : $\lnot > (\land \geq \lor) > (\to \geq \leftrightarrow)$(是否取等号看具体定义,默认相等)
   }%联接词结尾

  \subsection{命题公式的定义}{
    \begin{enumerate}
      \item 单个命题变元(或常元)($P,Q,R,...$)是命题公式.
      \item 如果$A$是命题公式,则$(\lnot A)$也是.
      \item 如果$A,B$都是命题公式,则$(A \land B),(A \lor B),(A \to B),(A \leftrightarrow B)$也是.
      \item 只有有限次应用$1 - 3$形成的符号串才是命题公式(有限长度).
    \end{enumerate}
  }%命题公式的定义结尾

  \subsection{命题公式举例}{
    $P$:简单命题(无连结词)

    $(\lnot(\lnot(\lnot P)))$, $\lnot\lnot\lnot P$:复合命题(有连结词)

    $(P \to (Q \to R))$, $P \to (Q \to R)$:括号不能乱去,此命题是"$P$"的结论是"如果$Q$则$R$".与代数运算类似,如果无括号就从左往右,如果有就优先结合括号.
  }%命题公式举例

  \subsection{真值表}{
    将公式可能的赋值全部列出,后面再列出公式的取值,就称为真值表.

    \begin{center}
      \begin{tabular}{c|c|c|c|c}
        \hline
        $P$                       & $Q$      & $P \to Q$                                   & $P \land\lnot P$                            & $P\land(P \lor Q) \leftrightarrow P$ \\
        \hline
        0                         & 0        & 1                                           & 0                                           & 1                                    \\
        0                         & 1        & 1                                           & 0                                           & 1                                    \\
        1                         & 0        & 0                                           & 0                                           & 1                                    \\
        1                         & 1        & 1                                           & 0                                           & 1                                    \\
        \hline
        \multicolumn{2}{c|}{赋值} & 可满足式 & $\underset{\mbox{矛盾式}}{\mbox{(永假式)}}$ & $\underset{\mbox{重言式}}{\mbox{(永真式)}}$
      \end{tabular}
    \end{center}

    \begin{itemize}
      \item 可满足式 : 至少有一种情况可以取值为真.
      \item 矛盾式 : 没有可以取真的情况.
      \item 永真式 : 所有的取值都取真,是可满足式的一种.
    \end{itemize}
  }%真值表结尾

 }%命题公式结尾

\section{等值演算}{

  \subsection{等值式}{
    等值式$A \Leftrightarrow B$的意思就是: $A \leftrightarrow B$是永真式 :

    \begin{center}
      \begin{tabular}{c|c|c|c|c}
        \hline
        $P$ & $Q$ & $P \to Q$ & $\lnot P \lor Q$ & $(P \to Q) \leftrightarrow (\lnot P \lor Q)$ \\
        \hline
        0   & 0   & 1         & 1                & 1                                            \\
        0   & 1   & 1         & 1                & 1                                            \\
        1   & 0   & 0         & 0                & 1                                            \\
        1   & 1   & 1         & 1                & 1                                            \\
        \hline
      \end{tabular}
    \end{center}

    $\therefore P \to Q \Leftrightarrow \lnot P \lor Q$

    意义就在于:证明是等值式后可以将两个公式互相随意替换而不改变运算结果.
  }%等值式结尾

  \subsection{基本等值式}{
    \begin{itemize}
      \item 幂等律 : $A \Leftrightarrow A \lor A,A \Leftrightarrow A \land A$
      \item 交换律 : $A \lor B \Leftrightarrow B \lor A,A \land B \Leftrightarrow B \land A$
      \item 结合律 : $(A \lor B) \lor C \Leftrightarrow A \lor (B \lor C),(A \land B) \land C \Leftrightarrow A \land (B \land C)$
      \item 分配律 : $A \land (B \lor C) \Leftrightarrow (A \land B) \lor (A \land C),A \lor (B \land C) \Leftrightarrow (A \lor B) \land (A \lor C)$
      \item 德$\cdot$摩根律 : $\lnot (A \lor B) \Leftrightarrow \lnot A \land \lnot B,\lnot (A \land B) \Leftrightarrow \lnot A \lor \lnot B$
      \item 吸收律 : $A \lor (A \land B) \Leftrightarrow A,A \land (A \lor B) \Leftrightarrow A$
      \item 零律 : $A \lor 1 \Leftrightarrow 1,A \land 0 \Leftrightarrow 0$
      \item 同一律 : $A \lor 0 \Leftrightarrow A,A \land 1 \Leftrightarrow A$
      \item 排中律 : $A \lor \lnot A \Leftrightarrow 1$
      \item 矛盾律 : $A \land \lnot A \Leftrightarrow 0$
      \item 双重否定律 : $\lnot \lnot A \Leftrightarrow A$
      \item 蕴涵等值式 : $A \to B \Leftrightarrow \lnot A \lor B$
      \item 等价等值式 : $A \leftrightarrow B \Leftrightarrow (A \to B) \land (B \to A)$
      \item 等价否定等值式 : $A \leftrightarrow B \Leftrightarrow \lnot A \leftrightarrow \lnot B$
      \item 假言易位 : $A \to B \Leftrightarrow \lnot B \to \lnot A$
      \item 归谬论 : $(A \to B) \land (A \to \lnot B) \Leftrightarrow \lnot A$
      \item 对偶原理 : 一个等值式$\lor - \land$互换,并将结果$0 - 1$互换,那么依然是对的.
      \item 排中律略有争议 : 在某些地方问题并不是非错即对,不过一般默认成立.
      \item 以上公式都可以通过列真值表证明.
    \end{itemize}
  }%基本等值式结尾

  \subsection{等值演算}{
    要证明等值式,除了列真值表还可以做等值演算.

    例 :

    \begin{math}
      P \to (Q \to R) \\
      \Leftrightarrow P \to (\lnot Q \lor R)(\mbox{蕴涵等值式,置换规则}) \\
      \Leftrightarrow \lnot P \lor (\lnot Q \lor R) \\
      \Leftrightarrow (\lnot P \lor \lnot Q) \lor R (\mbox{结合律}) \\
      \Leftrightarrow \lnot (P \land Q) \lor R (\mbox{德$\cdot$摩根律}) \\
      \Leftrightarrow (P \land Q) \to R (\mbox{蕴涵等值式})
    \end{math}
  }%等值演算结尾

 }%等值演算结尾

\section{命题推理逻辑}{

  \subsection{逻辑推理的形式结构}{
    前提 : $A_1.A_2,\dots,A_K$

    结论 : $B$

    推理的形式结构 : $(A_1 \land A_2 \land \dots \land A_K) \to B$
  }%推理的形式结构结尾

  \subsection{重要的推理定律}{
    推理定律-$A \Rightarrow B$(由A可以推出B) : $A \to B$是永真式

    \begin{itemize}
      \item \reasoningStructure{附加律}{$A \Rightarrow (A \lor B)$}{$A$}{$A \lor B$}{$A \to (A \lor B)$}
      \item \reasoningStructure{化简律}{$(A \land B) \Rightarrow A,(A \land B) \Rightarrow B$}{$A \land B$}{}{}
      \item \reasoningStructure{假言推理}{$(A \to B) \land A \Rightarrow B$}{$A \to B,A$}{$B$}{$((A \to B) \land A) \to B$}
      \item \reasoningStructure{拒取式}{$(A \to B) \land \lnot B \Rightarrow \lnot A$}{$A \to B,\lnot B$}{$\lnot A$}{$((A \to B) \land \lnot B) \to (\lnot A)$}
      \item \reasoningStructure{析取三段论}{$(A \lor B) \land \lnot A \Rightarrow B,(A \lor B) \land \lnot B \Rightarrow A$}{$A \lor B,\lnot A$}{$B$}{$((A \lor B)\land \lnot A) \to B$}
      \item \reasoningStructure{假言三段论}{$(A \to B) \land (B \to C) \Rightarrow (A \to C)$}{$A \to B,B \to C$}{$A \to C$}{$((A \to B) \land (B \to C)) \to (A \to C)$}
      \item \reasoningStructure{等价三段论}{$(A \leftrightarrow B) \land (B \leftrightarrow C) \Rightarrow (A \leftrightarrow C)$}{$A \leftrightarrow B,B \leftrightarrow C$}{$A \leftrightarrow C$}{$((A \leftrightarrow B) \land (B \leftrightarrow C)) \to (A \leftrightarrow C)$}
      \item \reasoningStructure{构造性两难}{$(A \to B) \land (C \to D) \land (A \lor C) \Rightarrow (B \lor D)$}{$A \to B,C \to D,A \lor C$}{$B \lor D$}{$(A \to B) \land (C \to D) \land (A \lor C) \to (B \lor D)$}
    \end{itemize}
  }%重要的推理定律结尾

  \subsection{判断推理正确的方法}{
    前提 : $P \to (Q \to R),P,Q$

    结论 : $R$

    方法一 : 推理的形式结构(判断是否为永真式).

    方法二 : 从前提推演结论(用前提一步步推结论).
  }%判断推理正确的方法结尾

 }%命题推理逻辑结尾

\section{一阶谓词逻辑}{
注释 : 谓词逻辑和命题逻辑不互通.在命题逻辑中最基本的单元是原子命题,无法细分.而在谓词逻辑中还要把命题作为陈述句分为"主语","谓语","宾语".

\subsection{个体}{
  将可以独立存在的课题(具体事务或抽象概念)称为个体或个体词,并用$a,b,c,\dots$表示个体常元,用$x,y,z,\dots$表示个体变元.(个体的函数还是个体,比如 : 设$a,b$是数,f(a,b)可以表示$a$和$b$的运算结果,比如$a + b$,$a \cdot b$等.)

  将个体变元的取值范围称为个体域,个体域可以是有穷或者无穷集合.人们称由宇宙间一切事物组成的个体域为全总个体域.

  通常来说如果没有特别说明那么默认使用全总个体域.
}%个体结尾

\subsection{谓词}{
  将表示个体性质或彼此之间关系的词称为谓词,常用$F,G,H,\dots$表示谓词常元或者谓词变元,用$F(x)$表示"$x$具有性质$F$",用$F(x,y)$表示"$x$和$y$具有关系$F$".

  例如,若$F(x)$表示"$x$是黑色的",$a$表示黑板,则$F(a)$表示"黑板是黑色的";

  如果$F(x,y)$表示"$x$大于$y$",则$F(5,2)$表示"5大于2".
}%谓词结尾

\subsection{量词,全称量词}{
  称表示数量的词为量词.他不是表示具体数量的词,而是表示范围.

  \begin{itemize}
    \item {
          全程量词 :

          全称量词是自然语言中的"所有的","一切的","任意的","每一个"."都"等的统称,用负号"$\forall$"表示.

          用$\forall x$表示个体域里的所有x;

          用$\forall x F(x)$表示个体域里所有的$x$都有性质$F$
          }
    \item {
          存在量词 :

          存在量词是自然语言中的"有一个","至少有一个","存在着","有的"等的统称,用负号"$\exists$"表示

          用$\exists x$表示存在个体域里的$x$;

          用$\exists x F(x)$表示在个体域里存在$x$具有性质$F$.
          }
  \end{itemize}
}%量词,全称量词结尾

\subsection{命题符号化}{
  一阶逻辑中命题符号化的两个基本公式 :

  \begin{itemize}
    \item {
          个体域中所有有性质$F$的个体都有性质$G$,应符号化为"$\forall x (F(x) \to G(x))$",其中 :

          $F(x)$ : $x$具有性质$F$,$G(x)$ : $x$具有性质$G$.
          }
    \item {
          个体域中存在着有兴致$F$同时有性质$G$的个体,应符号化为"$\exists x (F(x) \land G(x))$",其中 :

          $F(x)$ : $x$具有性质$F$,$G(x)$ : $x$具有性质$G$.
          }
  \end{itemize}

  例子 :
  \begin{enumerate}
    \item {
          人都吃饭 :

          令$F(x)$ : $x$是人,$G(x)$ : $x$吃饭.

          命题符号化为 : $\forall x (F(x) \to G(x))$.
          }
    \item {
          有人喜欢吃糖 :

          令$F(x)$ : $x$是人,$G(x)$ : $x$喜欢吃糖.

          命题符号化为 : $\exists x (F(x) \land G(x))$.
          }
    \item {
          男人都比女人跑得快(这是假命题).

          令$F(x)$ : $x$是男人,$G(y)$ : $y$是女人,$H(x,y)$ : x比y跑的快.

          命题符号化为 : $$
            \forall x (F(x) \to \forall y (G(y) \to H(x,y)))
          $$
          等价形式为 : $$
            \forall x \forall y (F(x) \land G(y) \to H(x,y))
          $$
          }
  \end{enumerate}
}%命题符号化

\subsection{一阶谓词逻辑公式}{
  一阶谓词逻辑公式也简称为公式,它的形成规则类似于命题逻辑公式,只需加上一条 : 若$A$是公式,则$\forall x A$与$\exists x A$也都是公式.

  在公式$\forall x A$和$\exists x A$中,称$x$为指导变元,称$A$为相应量词的辖域.在$\forall x$和$\exists x$的辖域中,$x$的所有出现都称为是约束出现,$A$中不是约束出现的变元称为自由出现.

  在一阶逻辑中,量词只能作用于个体上,不能作用于谓词上.

  在二阶逻辑中量词可以作用到谓词上.

  例如 : $$
    \forall x (F(x) \to \exists y (G(y) \land H(x,y,z)))
  $$

  其中$\forall x$的辖域为 : $(F(x) \to \exists y (G(y) \land H(x,y,z)))$,

  $\exists y$的辖域为 : $(G(y) \land H(x,y,z))$

  除了$z$是自由出现的变元以外,其他变元都是约束出现的. :
  $$
  \forall \textcolor{blue}{x} (F(\textcolor{blue}{x}) \to \exists y (G(y) \land H(\textcolor{blue}{x},y,z)))
  $$
  $$
  \forall x (F(x) \to \exists \textcolor{blue}{y} (G(\textcolor{blue}{y}) \land H(x,\textcolor{blue}{y},z)))
  $$
  }%一阶谓词逻辑公式结尾

  \subsection{解释}{
    对于给定的公式$A$,如果指定$A$的个体域为已知的$D$,并用特定的个体常元取代$A$中的个体常元,用特定函数取代$A$中的函数变元,用特定的谓词取代$A$中的谓词变元,则就构成了$A$的一个解释.

    给定的一个公式$A$可以有多种解释.

    例 : 给定公式$A$为$\forall x (F(x) \to G(x))$,有多种解释 :
    \begin{enumerate}
      \item {
            取个体域为实数集合,$F(x)$ : $x$是有理数,$G(x)$ : $x$能表示成分数 :

            $A$解释为 : "有理数都能表示成分数",这是真命题.
            }
      \item {
            取个体域为全总个体域,$F(x)$ : $x$是人,$G(x)$ : $x$长着黑头发 :

            $A$解释为 : "人都长着黑头发",这是假命题.
            }
    \end{enumerate}
  }%解释结尾

  \subsection{永真,永假,可满足,等值式}{
    \begin{itemize}
      \item 若$A$在任何解释下都为真,则称$A$为永真式.
      \item 若$A$在任何解释下都为假,则称$A$为永假式.
      \item 若$A$至少存在一个成真的解释,则称$A$为可满足式.
      \item 若$A \leftrightarrow B$是永真式,则称$A$与$B$是等值的,记为$A \Leftrightarrow B$,并称$A \Leftrightarrow B$为等值式.
    \end{itemize}
  }%永真,永假,可满足,等值式结尾

  \subsection{基本等值式}{
    \begin{itemize}
      \item {
            在有限个体域$D = {a_1,a_2,\dots,a_n}$中消去量词等值式 :

            \begin{enumerate}
              \item $\forall x A(x) \Leftrightarrow A(a_1) \land A(a_2) \land \dots \land A(a_n)$
              \item $\exists x A(x) \Leftrightarrow A(a_1) \lor A(a_2) \lor \dots \lor A(a_n)$
            \end{enumerate}
            }
      \item {
            量词否定等值式 :

            \begin{enumerate}
              \item $\lnot\forall x A(x) \Leftrightarrow \exists x \lnot A(x)$
              \item $\lnot\exists x A(x) \Leftrightarrow \forall x \lnot A(x)$
            \end{enumerate}
            }
      \item {
            量词辖域收缩与扩张等值式($B$中不含$x$) :

            \begin{enumerate}
              \item $\forall x (A(x) \lor B) \Leftrightarrow \forall x A(x) \lor B$
              \item $\forall x (A(x) \land B) \Leftrightarrow \forall x A(x) \land B$
              \item $\forall x (A(x) \to B) \Leftrightarrow \exists x A(x) \to B$
              \item $\forall x (B \to A(x)) \Leftrightarrow B \to \forall x A(x)$
              \item $\exists x (A(x) \lor B) \Leftrightarrow \exists x A(x) \lor B$
              \item $\exists x (A(x) \land B) \Leftrightarrow \exists x A(x) \land B$
              \item $\exists x (A(x) \to B) \Leftrightarrow \forall x A(x) \to B$
              \item $\exists x (B \to A(x)) \Leftrightarrow B \to \exists x A(x)$
            \end{enumerate}
            }
      \item {
            量词分配等值式 :

            \begin{enumerate}
              \item {
                    $\forall x (A(x) \land B(x)) \Leftrightarrow \forall x A(x) \land \forall x B(x)$

                    说明 : 全称量词对"$\land$"有分配律,但对"$\lor$"没有.
                    }
              \item {
                    $\exists x (A(x) \lor B(x)) \Leftrightarrow \exists x A(x) \lor \exists x B(x)$

                    说明 : 存在量词对"$\lor$"有分配率,但对"$\land$"没有.
                    }
            \end{enumerate}
            }
    \end{itemize}
  }%基本等值式结尾

  \subsection{前束范式}{
    若公式$A$具有形式$Q_1x_1Q_2x_2 \dots Q_kx_kB$,则称$A$为前束范式,其中$Q_i(1 \leq i \leq k)$为$\forall$或者$\exists$,$B$中不含量词.

    求前束范式时用基本等值式和换名规则.

    换名规则 : 将公式$A$中某量词辖域中出现的某个约束出现的个体变元及相应的指导变元$x_i$,都改成公式$A$中没有出现过的$x_j$,所得公式$A\derivative \Leftrightarrow A$

    例 :

    \begin{math}
      \forall x F(x) \lor \lnot\exists x G(x,y) \\
      \Leftrightarrow \forall x F(x) \lor \forall x \lnot G(x,y)\ \mbox{ (量词否定等值式)} \\
      \Leftrightarrow \forall x F(x) \lor \forall z \lnot G(z,y)\ \mbox{ (换名规则)} \\
      \Leftrightarrow \forall x (F(x) \lor \forall z \lnot G(z,y))\ \mbox{ (辖域扩张等值式)} \\
      \Leftrightarrow \forall x \forall z (F(x) \lor \lnot G(z,y))\ \mbox{ (辖域扩张等值式)} \\
      \Leftrightarrow \forall x \forall z (G(z,y) \to F(x))\ \mbox{ (蕴涵等值式)}
    \end{math}
  }%前束范式结尾

  \subsection{重要的推理定律}{
    \begin{enumerate}
      \item $\forall x A(x) \lor \forall x B(x) \Rightarrow \forall x (A(x) \lor B(x))$
      \item $\exists x (A(x) \land B(x)) \Rightarrow \exists x A(x) \land \exists x B(x)$
      \item $\forall x (A(x) \to B(x)) \Rightarrow \forall x A(x) \to \forall x B(x)$
      \item $\forall x (A(x) \to B(x)) \Rightarrow \exists x A(x) \to \exists x B(x)$
    \end{enumerate}

    说明 : 以上四条定律是单向的,所以在使用以上四条推理定律时,千万注意,别将他们当作等值式使用.这样会犯错误的.
  }%重要的推理定律

  }%一阶谓词逻辑结尾

  \section{充分必要条件}{

    充分必要条件(sufficient and necessary condition)简称为充要条件.

    在逻辑学中:
    \begin{itemize}
      \item 当命题"若P则Q"为真时,P称为Q的充分条件,Q称为P的必要条件.

            因此:

      \item 当命题"若P则Q"与"若Q则P"皆为真时,P是Q的充分必要条件,同时,Q也是P的充分必要条件.
      \item 当命题"若P则Q"为真,而"若Q则P为假时",P是Q的充分不必要条件,Q是P的必要不充分条件,反之亦然.
    \end{itemize}

    \subsection{必要条件}{
      P是Q的必要条件,代表"如果P是假,则Q是假"

      以逻辑符号表示:

      $\lnot P \to \lnot Q$

      通过否定后件,得出"如果Q是真,则P是真".

      $Q \to P$
    }%必要条件结尾

    \subsection{充分条件}{
      P是Q的充分条件,代表"如果P是真,则Q是真"或"如果Q是假,则P是假".

      以逻辑符号表示:

      $P \to Q$
    }%充分条件结尾

    \subsection{必要条件及充分条件}{
      P是Q的充分及必要条件,代表"当且仅当P是真,则Q是真".

      以逻辑符号表示:

      $P \leftrightarrow Q$

      留意$\lnot P \to \lnot Q$可以推出$Q \to P$.

      $(\lnot P \to \lnot Q) \land (P \to Q)$

      $(Q \to P) \land (P \to Q)$

      $P \leftrightarrow Q$
    }%必要条件及充分条件结尾

   }%充分必要条件结尾

}%逻辑学结尾