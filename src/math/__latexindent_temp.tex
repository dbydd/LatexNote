\chapter{数论}{

\section{整数的整除性}{

\subsection{带余除法}{
    设$n,m$都是整数且$n \neq 0$,则可以唯一的将$m$写作$m = q \cdot b + r$,其中$q,r$是整数,且$0 \leq r < \absoluteValue{n}$.$q$称为商(quotient),r称作余数(remainder),取余数称为取模,记作$r = \mod{m}{n}$

    \subsubsection{整除,倍数与约数}{
        \begin{itemize}
            \item 若余数$r = 0$,则称$m$能被$n$整除,或者$n$整除$m$,记作$n | m$
            \item 此时,称$m$是$n$的一个倍数(multiple),称$n$是$m$的一个约数或因子(divisor)
            \item 若$n | m$,则存在整数$q$使得$m = q \cdot n$,且有$n \leq \absoluteValue{m}$
        \end{itemize}
    }%整除,倍数与约数结尾

    \subsubsection{整除的线性组合,传递性与反对称性以及显然性质}{
        假设$a,b,c$是整除,$a \neq 0$,则 : 
        \begin{itemize}
            \item 若$a | b$且$a | c$,则对于任意的整数$x,y$,有$a | (xb + yc)$
            \item 若$b \neq 0: (a|b) \land (b | c) \Rightarrow a | c$
            \item 若$b \neq 0: (a | b) \land (b | a) \Rightarrow a = \pm b$
            \item 对于任意正整数$a$,有$a | a$和$1 | a$
        \end{itemize}
    }%整除的线性组合结尾

}

\subsection{素数与合数}{
若大于1的整数$p$的所有正因子只有$p$和$1$,则称其为质数或素数(prime);否则称其为合数(composite number)
 
\subsubsection{素数有无穷多个}{
    \begin{itemize}
        \item 假设只有有限个素数,设为$p_1, p_2, \dots, p_n$
        \item 令$m = p_1p_2...p_n + 1$,显然$m$无法被其中任意一个素数整除,因此矛盾:要么$m$本身是素数,要么存在大于$p_n$的素数可以整除$m$.
    \end{itemize}
}%素数有无穷多个结尾

\subsubsection{算数基本定理(唯一析因定理)}{
$$
\forall n > 1 : n = p_1^{k_1}p_2^{k_2} \dots p_s^{k_s}
$$

其中$p_1 < p_2 < \dots < p_s$是$s$个相异的素数,指数$k_i$都是正整数.

此表达式又称作整数$n$的素因子分解.
}%算数基本定理结尾

}%素数与合数结尾

\subsection{最大公因子与最小公倍数}{

    \subsubsection{最大公因子}{
        \begin{itemize}
            \item 设$a$和$b$是两个不全为0的整数,若整数$d$满足$d | a$且$d | b$,则称$d$是$a,b$的公因子(common divisor).
            \item 所有公因子中最大的称作$a$与$b$的最大公因子(greatest common divisor),记作$GCD(a,b)$.
            \item 若整数$a,b$的最大公因子为1,则称$a$与$b$互素(relatively prime).
        \end{itemize}

        若$a = p_1^{k_1}p_2^{k_2} \dots p_s^{k_s}$且$b = p_1^{l_1}p_2^{l_2} \dots p_s^{l_s}$,则 : $$
        GCD(a,b) = p_1^{min(k_1,l_1)}p_2^{min(k_2,l_2)} \dots p_s^{min(k_s,l_s)}
        $$
    }%最大公因子结尾

    \subsubsection{最小公倍数}{
        \begin{itemize}
            \item 设$a$和$b$是两个不全为0的整数,若整数$m$满足$a | m$且$b | m$,则称$m$是$a,b$的公倍数.
            \item 所有公倍数中最小的正整数称为$a$与$b$的最小公倍数(least common multiple),记作$LCM(a,b)$.
        \end{itemize}

        若$a = p_1^{k_1}p_2^{k_2} \dots p_s^{k_s}$且$b = p_1^{l_1}p_2^{l_2} \dots p_s^{l_s}$,则 : $$
        LCM(a,b) = p_1^{max(k_1,l_1)}p_2^{max(k_2,l_2)} \dots p_s^{max(k_s,l_s)}
        $$
    }%最小公倍数结尾

    显然,对于任意的正整数$a$ : 
    \begin{itemize}
        \item $GCD(0,a) = a$
        \item $GCD(1,a) = 1$
        \item $LCM(1,a) = a$
    \end{itemize}

\subsubsection{欧几里得算法(辗转相除法)}{
    设$a = qb + r,$其中$a,b,q,r$都是整数,则 : $$
    GCD(a,b) = GCD(b,r)
    $$

    \begin{itemize}
        \item 若$d | a$且$d | b$,则有$d | b$且$d | r = (a - qb)$
        \item 若$d | b$且$d | r$,则有$d | (qb + r)$,即$d | a$.
        \item 于是$a$与$b$的公因子集合和$b$与$r$的公因子集合相同.继而最大公因子集合相同.
    \end{itemize}    

    基于这种性质,提出了欧几里得算法(辗转相除法),这种算法可以求两个数之间的最大公因子,以下为代码实现(c语言) : 
    \begin{lstlisting}
int GCD(int a,int b){
    if(b == 0){
        return a;
    } else { 
        return GCD(b,a%b);
    }
}
    \end{lstlisting}
}%辗转相除法结尾

\subsubsection{裴蜀等式}{
    由辗转相除法又可以导出此定理 : 
    \begin{itemize}
        \item 对于不全为0的整数$a,b,d$,方程$sa + tb = d$存在整数解$s,t$当且仅当$GCD(a,b) | d$.
        \item $sa + tb = d$称为裴蜀(Bezout)等式或贝祖等式.
    \end{itemize}
}%裴蜀等式结尾

}%最大公因子与最小公倍数结尾

}%整数的整除性结尾

 }%数论结尾